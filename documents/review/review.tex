\documentclass{article}

\usepackage[ngerman]{babel}
\usepackage[utf8]{inputenc}
\usepackage[T1]{fontenc}
\usepackage{hyperref}
\usepackage{csquotes}

\usepackage[
    backend=biber,
    style=apa,
    sortlocale=de_DE,
    natbib=true,
    url=false,
    doi=false,
    sortcites=true,
    sorting=nyt,
    isbn=false,
    hyperref=true,
    backref=false,
    giveninits=false,
    eprint=false]{biblatex}
\addbibresource{../references/bibliography.bib}

\title{Review des Papers "Die Ki in der Justiz" von Shania Müller}
\author{Noah Cooper}
\date{\today}

\begin{document}
\maketitle

\abstract{
    Dies ist ein Review der Arbeit zum Thema Verwendung von künstlicher Intelligenz in der Justiz von Shania Müller.
}

\tableofcontents
\newpage

\section{Einleitung}
    Die Einleitung gibt einen guten und kurzen ersten Einblick in die Arbeit und macht Lust darauf, weiter zu lesen.

\section{KI vor Gericht und COMPAS}
    Die Verwendung der KI im Gericht ist sehr klar und verständlich erklärt, ebenso wie das Tool COMPAS funktioniert.
    \newline Nach dem Lesen frage ich mich aber, ob es neben COMPAS noch andere ähnliche oder verschiedenartige Tools gibt. 

\section{KI-Training}
    Das Training der KI wurde simpel und verständlich erläutert und ich habe nichts hinzuzufügen oder abzuändern. Ich finde es gut, dass Shania am Ende des Abschnittes noch hervorgehoben hat, worauf sich die Entwickler der KI beim Training besonders achten.

\section{Menschen vor Gericht wegen KI}
    Ich finde diesen Fall ein sehr geeignetes Beispiel für dieses Thema und Shania hat ihn sehr gut dargestellt, indem sie die Kritik an der Verwendung der KI aber auch die positiven Aspekte aufgezeigt hat.
    \newline Was man noch verbessern könnte, wäre, gleich am Anfang des Falls mehr Kontext zu bieten (z.B. Alter des Angeklagten, Ort des Geschehens), damit sich der/die Leser*in ein besseres Bild des Vorfalls machen kann.

\section{Ethischer Zusammenhang}
    Der Abschnitt zeigt deutlich auf, dass der wichtigste Aspekt der KI in der Justiz die Transparenz ihrer Arbeitsweise ist, und dass man sehr präzise und exakt mit ihr arbeiten muss. Shania sagt klar, dass es wichtig ist zu wissen, bei wem die Verantwortung der KI-Entscheidungen liegt.

\section{Fazit der Review}
    Ich finde Shania's Paper sehr ordentlich strukturiert und einfach zu lesen und zu verstehen. Sie hat viele verschiedene Facetten der KI in der Justiz angesprochen und erklärt, wobei die Transparenz immer den roten Faden gebildet hat.
    \newline Es wäre noch spannend gewesen, von einem Fall zu lesen, bei dem ein*e Angeklagte*r durch die KI zu Schaden gekommen ist.

\printbibliography

\end{document}
