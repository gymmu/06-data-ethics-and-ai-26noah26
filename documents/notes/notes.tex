\documentclass{article}

\usepackage[ngerman]{babel}
\usepackage[utf8]{inputenc}
\usepackage[T1]{fontenc}
\usepackage{hyperref}
\usepackage{csquotes}

\usepackage[
    backend=biber,
    style=apa,
    sortlocale=de_DE,
    natbib=true,
    url=false,
    doi=false,
    sortcites=true,
    sorting=nyt,
    isbn=false,
    hyperref=true,
    backref=false,
    giveninits=false,
    eprint=false]{biblatex}
\addbibresource{../references/bibliography.bib}

\title{Notizen zum Projekt Data Ethics}
\author{Noah Cooper}
\date{\today}

\begin{document}
\maketitle

\abstract{
    Dieses Dokument ist eine Sammlung von Notizen zu dem Projekt. Die Struktur innerhalb des
    Projektes ist gleich ausgelegt wie in der Hauptarbeit, somit kann hier einfach geschrieben
    werden, und die Teile die man verwenden möchte, kann man direkt in die Hauptdatei ziehen.
}

\tableofcontents

\section{
    Leifrage: Ist es ethisch vertretbar, KI für die Erstellung von Kunst zu verwenden?
}

\section{Meine Meinung vor der Recherche:}
    Mit der KI lassen sich viele interessante Kunstwerke erstellen, wie man anhand von Ai Image 
    Generators erkennen kann. Ich kann mir aber vorstellen, dass viele Künstler*innen eine Gefahr 
    darin sehen, dass ihre Arbeitsplätze und ihre Einnahmequelle verschwinden könnten. Die Kunst der KI 
    hat in meinen Augen ohne den Aspekt der menschlichen Überlegungen nicht so viel Wert. Die KI denkt 
    sich nicht aus, welche Nachricht sie mit ihrer Kunst teilen will, weshalb man sie auch nicht 
    interpretieren und darüber diskutieren kann.


\section{Wie wird KI trainiert?}

\section{Was sagen Befürworter*innen der KI-Kunst?}


\input{section_ai.tex}

\printbibliography

\end{document}
