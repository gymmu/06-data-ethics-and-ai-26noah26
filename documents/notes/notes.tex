\documentclass{article}

\usepackage[ngerman]{babel}
\usepackage[utf8]{inputenc}
\usepackage[T1]{fontenc}
\usepackage{hyperref}
\usepackage{csquotes}

\usepackage[
    backend=biber,
    style=apa,
    sortlocale=de_DE,
    natbib=true,
    url=false,
    doi=false,
    sortcites=true,
    sorting=nyt,
    isbn=false,
    hyperref=true,
    backref=false,
    giveninits=false,
    eprint=false]{biblatex}
\addbibresource{../references/bibliography.bib}

\title{Notizen zum Projekt Data Ethics}
\author{Name des Autors}
\date{\today}

\begin{document}
\maketitle

\abstract{
    Dieses Dokument ist eine Sammlung von Notizen zu dem Projekt. Die Struktur innerhalb des
    Projektes ist gleich ausgelegt wie in der Hauptarbeit, somit kann hier einfach geschrieben
    werden, und die Teile die man verwenden möchte, kann man direkt in die Hauptdatei ziehen.
}

\tableofcontents

\section{Leitfrage
Leifrage: Ist es ethisch vertretbar, KI für die Herstellung von Kunst zu verwenden?
}


\input{section_ai.tex}

\printbibliography

\end{document}
